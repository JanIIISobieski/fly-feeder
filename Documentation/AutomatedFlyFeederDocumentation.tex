\documentclass[12pt]{article}

\usepackage[top=0.7in, left=0.7in, right=0.7in, bottom=0.7in]{geometry}
\usepackage{amsmath, amssymb, gensymb}
\usepackage{graphicx}
\usepackage[font=footnotesize, labelfont=bf]{caption}
\usepackage{subcaption}
\usepackage{float}
\usepackage{tabularx}
\usepackage{booktabs}
\usepackage{multicol}
\usepackage{multirow}
\usepackage{epstopdf}
\usepackage{appendix}
\usepackage{wrapfig}
\usepackage{paralist}
\usepackage{enumitem}
\usepackage{cancel}
\usepackage[english]{babel, fancyref}
\usepackage{hyperref}
\usepackage{xfrac}
\usepackage{lscape}
\usepackage{longtable}
\usepackage{relsize}
\usepackage{fancyhdr}
\usepackage{blindtext}
\usepackage{pdfpages}
\usepackage[utf8]{inputenc}
\usepackage{cite}
\usepackage{microtype}
\bibliographystyle{abbrv}

\lhead{}
\chead{}
\rhead{}

\pagestyle{plain}

% Allow for line breaking in tables
\newcolumntype{L}[1]{>{\let\newline\\\arraybackslash\hspace{0pt}}m{#1}}					% jusitified column
\newcolumntype{C}[1]{>{\centering\let\newline\\\arraybackslash\hspace{0pt}}m{#1}}		% centered column
\newcolumntype{R}[1]{>{\raggedleft\let\newline\\\arraybackslash\hspace{0pt}}m{#1}}		% right-aligned column

\floatstyle{plaintop}
\restylefloat{table}


\newcommand{\divider}{\vspace*{-4mm} \rule{\linewidth}{1pt} \vspace{-4pt} }

\fancyrefchangeprefix{\fancyreftablabelprefix}{tbl}
\fancyrefchangeprefix{\fancyrefeqlabelprefix}{eqn}

\title{Automated Fly Feeder Documentation}
\author{Gabriel Antoniak}
\date{\today}

\graphicspath{{./Diagrams/}}

\begin{document}
\maketitle
\divider

\section{Equipment List}
\renewcommand{\arraystretch}{1.2}
\begin{table}[H]
	\centering
	\begin{tabular}{ r | l }
		\textbf{Item}  & \textbf{Amount} \\ \cline{1-2}
		\href{https://store.arduino.cc/usa/arduino-mega-2560-rev3}{Arduino Mega}   &  1\\
		\href{https://www.sparkfun.com/products/9806}{10K Trimpot}    &  2\\
		\href{https://www.sparkfun.com/products/241}{IR Emitter}     &  2\\
		\href{https://www.sparkfun.com/products/241}{IR Sensor}      &  2\\
		\href{https://www.sparkfun.com/products/8982}{1 mF Capacitor} &  1\\
		\href{https://www.sparkfun.com/products/14490}{330 $\Omega$ Resistor} &  2\\ 
		\href{https://www.sparkfun.com/products/14491}{10 k$\Omega$ Resistor} &  2\\
		\href{https://www.pololu.com/product/1200}{Stepper Motor} & 1\\
		\href{https://www.sparkfun.com/products/9456}{LM358N Op-Amp} & 2\\
		\href{https://www.pololu.com/product/2858}{5V Regulator} & 1\\
		\href{https://www.pololu.com/product/2855}{12V Regulator} & 1\\
		\href{https://www.actuonix.com/L12-I-Micro-Linear-Actuator-Internal-Controller-p/l12-i.htm}{Linear Actuator} & 2\\
		\href{https://www.circuitspecialists.com/cw230.html}{Stepper Motor Driver} & 1\\
		\href{https://www.pololu.com/product/2485}{5V Relay}  & 2\\ \bottomrule
	\end{tabular}
	\caption{Necessary equipment and respective links to the product}
	\label{tbl:equipment}
\end{table}

For specific information regarding individual components, see the specification sheets in \texttt{SpecSheets/} folder in the same directory as this .pdf.

\section{Software}
The software required for the design of the project:
\begin{itemize}
	\item \href{https://www.autodesk.com/education/free-software/featured}{Autodesk Inventor Professional (2019 or onwards)} for the mechanical design
	\item \href{https://www.autodesk.com/education/free-software/featured}{EAGLE} for the design of the printed circuit boards (PCBs)
	\item Adobe Illustrator (Vikas will know how to obtain this) for opening and editing .DXF files exported from Inventor to ensure the PCBs will be of the right shape
\end{itemize}

The software required for rapid prototyping of the design:
\begin{itemize}
	\item \href{www.3dprinteros.com}{3DPrinterOS} (or \href{https://ultimaker.com/en/products/ultimaker-cura-software}{CURA}, if local 3D printer)
	\item \href{https://support.voltera.io/hc/en-us/articles/115002633033-EAGLE-Export-Guide}{Voltera V-1 CAM (Computer-Aided Manufacturing) plug-in} for EAGLE
	\item \href{https://support.voltera.io/hc/en-us/articles/115002633033-EAGLE-Export-Guide}{Voltera design help plug-in} for EAGLE
	\item \href{https://www.inventables.com/technologies/easel}{Easel} for Carvey CNC machine in case small parts need to be machined
	\item \href{https://oshpark.com/}{OSHPark} for manufacturing of PCBs -- this might be easier and faster than using a Voltera-V1, EAGLE has a CAM processor\footnote{A CAM Processor converts the board file into actual code for machines for their manufacture -- how to drill, where to place solder, pads, leads, vias, etc.} for exporting a .zip file specifically for OSHPark
\end{itemize}

\section{Part Notes}

\subsection{Stepper Motor}
\begin{itemize}
	\item At 200 steps per rotation, with 8 flies, that is 25 steps to get to the next fly
	\item The current limit of 1.2 A should not be exceeded
	\item Black-Yellow-Green are all connected together in one solenoid. Since the stepper driver is coded to drive it as a bipolar motor, the yellow lead is unnecessary. As such, only black and green are connected to the stepper driver.
	\item Red-White-Blue are all connected together in the other solenoid. Similarly as above, only the red and blue wires are connected to the stepper driver, and the white lead is unnecessary.
	\item Motor must be firmly attached to plate minimize vibrations, as these vibrations can be quite loud	
\end{itemize}

\subsection{Arduino Mega}
\begin{itemize}
	\item Only some pins can function as digital interrupts. These pins are: 2, 3, 18, 19, 20, 21.
	\item Each I/O pin should deliver no more than 40 mA of current. Any more risks damage to the board. The total I/O current should not exceed 200 mA.
	\item The total current from the 5 V pin on the Arduino should not exceed 500 mA.
\end{itemize}

\subsection{IR Emitters/Sensors}

\begin{itemize}
	\item The emitters are labeled with a yellow dot, while the sensors with a red dot
	\item The voltage measured by the Arduino or sent to the comparator should be measured after the 10 k$\Omega$ resistor but before the phototransistor (sensor)
	\item A resistor to lower the voltage and current to the IR diode should be used to prevent frying the component. A 330 $\Omega$ resistor is sufficient.
	\begin{table}[H]
		\centering
		\begin{tabular}{c | c | c}
			\textbf{State} 	& \textbf{Analog} & \textbf{Voltage} (V) \\ \cline{1-3}
			Fully blocked 	& 800  & 3.91\\
			Fly blocking	& 100  & 0.49\\
			Unbroken 		& 30   & 0.15\\ 
		\end{tabular}
		\caption{Approximate values for the IR beams when powered with 5 V}
		\label{tbl:irstates}
	\end{table}
\end{itemize}

\subsection{Actuonix Linear Actuator}
\begin{itemize}
	\item Blue wire -- voltage input signal (0 to 5 V)
	\item Purple wire -- feedback signal (0 to 3.3 V)
	\item Red wire -- power (12 V)
	\item Black wire -- ground
	\item The position signal should be between 0.1 V to 4.9 V to avoid the motor stalling at either max extension or max retraction, which can damage the motor
	\item The position signal is sent as a pulse width modulated wave: i.e. the voltage is proportional to the duty cycle of a square wave.
\end{itemize}

\section{Coding}
This one might be a matter of preference, as the Arduino IDE itself is sufficient for programming. However, it does not come with a lot of features that useful for any coding experience such as custom syntax highlighting, auto-complete, debugging tools, variable tracking, etc. \href{https://www.visualmicro.com/}{Visual Micro} plug-in for Microsoft Visual Studio is by far the better choice, makes programming significantly more straightforward and streamlined.

\subsection{Serial communication codes}
\begin{table}[H]
	\centering
	\begin{tabular}{c l}
		\texttt{'b'}  & begin experiment\\
	    \texttt{'f'}  & fly passed signal\\
		\texttt{'e'}  & end of experimental trial\\
		\texttt{'c'}  & completion of all trials\\				
		\texttt{'\%d'} & digit corresponding to fly tube\\
	\end{tabular}
	\caption{Codes sent by Arduino and MATLAB over serial to inform device status}
	\label{tbl:arduino_codes}
\end{table}

\subsection{Wire connections to Arduino}

\begin{table}[H]
	\centering
	\begin{tabular}{c c}
		\multicolumn{2}{c}{\textbf{Align fly cartridge}}\\ \midrule
		POT pin on aligner PCB & A0 \\
		RAW pin on aligner PCB & A1 \\
		DIG pin on aligner PCB & 3$^*$ \\ \midrule
		\multicolumn{2}{c}{\textbf{Fly passing}}\\ \midrule
		POT pin for fly passing & A6 \\
		RAW pin for fly passing & A7 \\
		DIG pin for fly passing PCB & 2$^*$ \\ \midrule
		\multicolumn{2}{c}{\textbf{Stepper motor}}\\ \midrule
		STEP pin (CP+) & 12 \\
		DIR pin (CW+) & 13 \\ \midrule
		\multicolumn{2}{c}{\textbf{Linear actuator}}\\ \midrule
		Motor 1 Feedback & A3 \\
		Motor 2 Feedback & A4 \\
		Motor 3 Feedback (Optional) & A5 \\
		Motor 1 Setter & 4 \\
		Motor 2 Setter & 5\\		
		Motor 3 Setter (Optional) & 6\\ \midrule
		\multicolumn{2}{c}{\textbf{Solenoids}}\\ \midrule
		Solenoid A & 7 \\
		Solenoid B & 8 \\
		Solenoid C (Optional) & 9 \\
		Solenoid D (Optional) & 10\\
	\end{tabular}
	\caption{The pin assignments for the device. $^*$ refers to the fact the pin is used for its interrupt capabilities}
\end{table}

\section{Circuit Diagram}
\includepdf[landscape=true]{./Diagrams/CircuitDiagram.pdf}

\end{document}
